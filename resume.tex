% !TEX program = xelatex

\documentclass{resume}

\begin{document}
\pagenumbering{gobble}

\name{Bojin ZHAO}

\basicInfo{
  \email{bo.bojinzhao@gmail.com} \textperiodcentered\
  \phone{(+86) 166-7565-3415} \textperiodcentered\
  \homepage[Google Scholar]{https://scholar.google.com/citations?user=eWqpGUkAAAAJ}
}

\section{\faGraduationCap\ Education}
\datedsubsection{\textbf{University of Macau (UM)}, Macau, China}{2021 -- 2025}
\textit{Ph.D. in Biomedical Sciences}\\
Research direction: Computational Biology, Deep Learning, Multi-omics

\datedsubsection{\textbf{The University of Edinburgh (UoE)}, Edinburgh, UK}{2019 -- 2020}
\textit{M.Sc. in Bioinformatics}\\
Focus: Biological Sciences, Information Science

\datedsubsection{\textbf{University of Macau (UM)}, Macau, China}{2015 -- 2019}
\textit{B.Sc. in Biomedical Sciences}\\
Core courses: Molecular Biology, Statistics, Data Science, Information Systems

\section{\faUsers\ Experience \& Major Projects}

\datedsubsection{\textbf{DDR-Origins: Evolutionary Analysis of Pathogenic Variants in DDR Genes}}{2021 -- 2023}
\role{Computational Biology, Phylogenetics, Ancient DNA}{}
\textbf{Project Overview:}\\
Integrated modern humans, ancient humans, and 100-vertebrate genomes to investigate evolutionary
origins of pathogenic variants in DNA Damage Repair (DDR) genes.

\textbf{Responsibilities:}
\begin{itemize}
  \item Performed cross-species sequence alignment, phylogenetic tree construction, and dN/dS
        selection inference using \textbf{PHAST, PAML, MEGA}.
  \item Processed \textbf{5000+ ancient human genomes}: BAM cleaning, variant calling,
        pseudo-mutation correction (mapDamage), and functional annotation.
  \item Built LAMP-based database (Linux, Apache, MySQL, PHP) for interactive visualization,
        online query, and multi-dimensional result comparison.
  \item Integrated Python (Pandas, NumPy), SAMtools, BCFtools, ANNOVAR for scalable data analysis.
\end{itemize}

\textbf{Outcome:}\\
Constructed a complete evolutionary pipeline and public database:\\
\texttt{https://genemutation.boboz.io/dbDDR-AncientHumans}

% -----------------------------------------------------

\datedsubsection{\textbf{DepGPS: Deep Graph Model for Cancer Gene Dependency Prediction}}{2023 -- 2025}
\role{Deep Learning, GNN, Transformer, Multi-omics Integration}{}
\textbf{Project Overview:}\\
Developed a deep learning framework combining GNN and Transformer for predicting cancer gene
dependencies using multi-omics data (expression, mutation, CNV) and PPI networks.

\textbf{Responsibilities:}
\begin{itemize}
  \item Designed a GNN + Transformer hybrid architecture for cross-level representation learning.
  \item Built embedding pipeline: Gaussian-mixture-based expression discretization, Gene2Vec
        embedding, and PPI topology/structure embedding.
  \item Conducted GPU-based training (NVIDIA), hyperparameter tuning, and context-dependent
        dependency inference and perturbation simulation.
\end{itemize}

\textbf{Outcome:}\\
Achieved prediction correlations of \textbf{0.94 (training)} and \textbf{0.91 (testing)}, outperforming
traditional ML and single deep models; demonstrated utility in synthetic lethality discovery
and drug target prioritization.

% -----------------------------------------------------

\datedsubsection{\textbf{Asian \textit{BRCA} Variation: Ethnic-specificity, Evolution Origin \& Clinical Impact}}{2021 -- 2022}
\role{Population Genomics, Database Construction, Cancer Genetics}{}
\textbf{Project Overview:}\\
Performed a systematic analysis of \textit{BRCA1}/\textit{BRCA2} germline variants in Asian populations, integrating literature mining, database curation, population-level comparison, evolutionary tracing, and clinical significance evaluation.

\textbf{Responsibilities:}
\begin{itemize}
  \item Curated and standardized \textbf{7587 \textit{BRCA} variants} from \textbf{685,000+} Asian individuals across 40 countries; conducted HGVS normalization, annotation and ACMG classification.
  \item Built the open-access \textbf{dbBRCA-Asian} database using \textbf{LAMP (Linux, Apache, MySQL, PHP)} for online \textit{BRCA} variant query, visualization and population comparison.
  \item Performed comparative population genetics: ethnic-specificity assessment across Asian regions; cross-continental comparison with Latin American, Middle Eastern and European datasets.
  \item Investigated variant evolutionary origin using \textbf{4800+ ancient human genomes} and Neanderthal/Denisovan genomes to estimate arising time and ancestral distribution.
  \item Applied structure-based deleteriousness evaluation (Ramachandran Plot–Molecular Dynamics Simulation) to interpret ~500 unknown \textit{BRCA} missense variants.
\end{itemize}

\textbf{Outcome:}\\
Generated the largest Asian \textit{BRCA} variant dataset to date (7587 variants), revealed strong ethnic-specific patterns and evolutionary origins, and established the \texttt{https://genemutation.boboz.io/Asian-BRCA/} database for clinical and research use.

% -----------------------------------------------------

\section{\faBook\ Publications (selected)}
For full list: Google Scholar
\begin{itemize}
  \item \textbf{Zhao B}, Li J, Sinha S, Qin Z, Kou SH, Xiao F, Lei H, Chen T, Cao W, Ding X, Wang SM*.\\
        \textbf{Pathogenic variants in human DNA damage repair genes mostly arose in recent human history.}\\
        \textit{BMC Cancer}, 2024.
  \item Qin Z\#, Li J\#, Tam B\#, Siddharth S\#, \textbf{Zhao B\#}, Bhaskaran SP, Huang T, Wu X, Chian JS, Guo M,
        Kou SH, Lei H, Zhang L, Wang X, Lagniton PNP, Xiao F, Jiang X, Wang SM*.\\
        \textbf{Ethnic-specificity, evolution origin and deleteriousness of Asian \textit{BRCA} variation revealed by over 7500 \textit{BRCA} variants derived from Asian population.}\\
        \textit{Int J Cancer}, 2023.
  \item Fu H\#, \textbf{Zhao B\#}, Wang P*.\\
        \textbf{Modeling cancer dependency with deep graph models.}\\
        \textit{bioRxiv}, 2024.
  \item Tam B, Qin Z, ..., \textbf{Zhao B}, Sinha S, Lei CL, Wang SM*.\\
        \textbf{Comprehensive classification of TP53 somatic missense variants based on their impact on p53 structural stability.}\\
        \textit{Brief Bioinform}, 2024.
\end{itemize}

\section{\faCogs\ Skills \& Interests}
\begin{itemize}[parsep=0.5ex]
  \item \textbf{Languages:} Mandarin (Native), English (Fluent; IELTS Speaking 7.0)
  \item \textbf{Programming \& Scripting:} Shell, Python, R, PHP, SQL, HTML/CSS, JavaScript, \LaTeX
  \item \textbf{Frameworks \& Libraries:} Biopython, PyTorch, PyTorch Geometric, NetworkX, NumPy,
        Pandas, scikit-bio, gseapy, scanpy, scvi-tools
  \item \textbf{Software:} Docker, Conda, Adobe Illustrator, Adobe Photoshop
  \item \textbf{Interests:} Piano, Snooker
\end{itemize}

\section{\faFlask\ Professional Interests}
Multi-Omics Integration · AI-Driven Computational Biology · Graph Neural Networks · Biological LLM-Based Agents

\end{document}
