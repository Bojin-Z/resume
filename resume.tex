% !TEX program = xelatex

\documentclass{resume}

\begin{document}
\pagenumbering{gobble}

\name{Bojin ZHAO}

\basicInfo{
  \email{bo.bojinzhao@gmail.com} \textperiodcentered\
  \phone{(+86) 166-7565-3415} \textperiodcentered\
  \homepage[Google Scholar]{https://scholar.google.com/citations?user=eWqpGUkAAAAJ}
}

\section{\faGraduationCap\ Education}
\datedsubsection{\textbf{University of Macau (UM)}, Macau, China}{2021 -- 2025}
\textit{Ph.D. in Biomedical Sciences}\\
Research direction: Computational Biology, Deep Learning, Multi-omics

\datedsubsection{\textbf{The University of Edinburgh (UoE)}, Edinburgh, UK}{2019 -- 2020}
\textit{M.Sc. in Bioinformatics}\\
Focus: Biological Sciences, Information Science

\datedsubsection{\textbf{University of Macau (UM)}, Macau, China}{2015 -- 2019}
\textit{B.Sc. in Biomedical Sciences}\\
Core courses: Molecular Biology, Statistics, Data Science, Information Systems

\section{\faUsers\ Experience \& Major Projects}

\datedsubsection{\textbf{DDR-Origins: Evolutionary Analysis of Pathogenic Variants in DDR Genes}}{2021 -- 2023}
\role{Computational Biology, Phylogenetics, Ancient DNA}{}
\textbf{Project Overview:}\\
Integrated modern humans, ancient humans, and 100-vertebrate genomes to investigate evolutionary
origins of pathogenic variants in DNA Damage Repair (DDR) genes.

\textbf{Responsibilities:}
\begin{itemize}
  \item Orchestrated a high-throughput comparative genomics pipeline combining multiple sequence alignment (\textbf{MAFFT, Muscle}) and phylogenetic inference (\textbf{RAxML, IQ-TREE}).
  \item Applied Maximum Likelihood and Bayesian inference methods using \textbf{PAML (codeml)} and \textbf{PHAST} to detect lineage-specific dN/dS selection pressures.
  \item Processed raw sequencing data from \textbf{5000+ ancient human genomes}: implemented \textbf{mapDamage} for post-mortem damage pattern analysis, and utilized \textbf{SAMtools/BCFtools} for precise variant calling and filtering.
  \item Developed a LAMP-stack database (Linux, Apache, MySQL, PHP) incorporating \textbf{JBrowse} for visualization, enabling interactive queries of variant evolutionary trajectories.
\end{itemize}

\textbf{Outcome:}\\
Constructed a complete evolutionary pipeline and public database:\\
\texttt{https://genemutation.boboz.io/dbDDR-AncientHumans}

% -----------------------------------------------------

\datedsubsection{\textbf{DepGPS: Deep Graph Model for Cancer Gene Dependency Prediction}}{2023 -- 2025}
\role{Deep Learning, GNN, Transformer, Multi-omics Integration}{}
\textbf{Project Overview:}\\
Developed a deep learning framework combining GNN and Transformer for predicting cancer gene
dependencies using multi-omics data (expression, mutation, CNV) and PPI networks.

\textbf{Responsibilities:}
\begin{itemize}
  \item Architected a hybrid deep learning model integrating \textbf{Graph Convolutional Networks (GCN)} for PPI topology and \textbf{Multi-Head Self-Attention (Transformer)} for feature context extraction.
  \item Engineered a robust feature embedding system: applied \textbf{Gaussian Mixture Models (GMM)} for expression discretization and \textbf{Gene2Vec} for dense vector representation of biological entities.
  \item Optimized model performance via extensive hyperparameter tuning and ablation studies on NVIDIA GPUs, utilizing \textbf{PyTorch Geometric} for graph message passing efficiency.
  \item Implemented customized loss functions to handle data sparsity in synthetic lethality prediction.
\end{itemize}

\textbf{Outcome:}\\
Achieved prediction correlations of \textbf{0.94 (training)} and \textbf{0.91 (testing)}, outperforming
traditional ML (Random Forest, SVM) and single deep models.

% -----------------------------------------------------

\datedsubsection{\textbf{Asian \textit{BRCA} Variation: Ethnic-specificity, Evolution Origin \& Clinical Impact}}{2021 -- 2022}
\role{Population Genomics, Database Construction, Cancer Genetics}{}
\textbf{Project Overview:}\\
Performed a systematic analysis of \textit{BRCA1}/\textit{BRCA2} germline variants in Asian populations, integrating literature mining, database curation, population-level comparison, evolutionary tracing, and clinical significance evaluation.

\textbf{Responsibilities:}
\begin{itemize}
  \item Constructed a standardized variant processing workflow using \textbf{VEP (Variant Effect Predictor)} and \textbf{ANNOVAR} for functional annotation across \textbf{685,000+ samples}.
  \item Conducted population structure analysis using Principal Component Analysis (\textbf{PCA}) and admixture modeling to quantify ethnic specificity across 40 Asian countries.
  \item Traced variant evolutionary origins by cross-referencing \textbf{PLINK} formatted genotype data with 4800+ ancient genomes and Neanderthal/Denisovan references.
  \item Assessed protein structural stability for VUS (Variants of Uncertain Significance) using \textbf{Molecular Dynamics (MD) simulations} and Ramachandran plots.
\end{itemize}

\textbf{Outcome:}\\
Generated the largest Asian \textit{BRCA} variant dataset to date (7587 variants) and established the \texttt{https://genemutation.boboz.io/Asian-BRCA/} database.

% -----------------------------------------------------

\section{\faBook\ Publications (selected)}
For full list: Google Scholar
\begin{itemize}
  \item \textbf{Zhao B}, Li J, Sinha S, Qin Z, Kou SH, Xiao F, Lei H, Chen T, Cao W, Ding X, Wang SM*.\\
        \textbf{Pathogenic variants in human DNA damage repair genes mostly arose in recent human history.}\\
        \textit{BMC Cancer}, 2024.
  \item Qin Z\#, Li J\#, Tam B\#, Siddharth S\#, \textbf{Zhao B\#}, Bhaskaran SP, Huang T, Wu X, Chian JS, Guo M,
        Kou SH, Lei H, Zhang L, Wang X, Lagniton PNP, Xiao F, Jiang X, Wang SM*.\\
        \textbf{Ethnic-specificity, evolution origin and deleteriousness of Asian \textit{BRCA} variation revealed by over 7500 \textit{BRCA} variants derived from Asian population.}\\
        \textit{Int J Cancer}, 2023.
  \item Fu H\#, \textbf{Zhao B\#}, Wang P*.\\
        \textbf{Modeling cancer dependency with deep graph models.}\\
        \textit{bioRxiv}, 2024.
\end{itemize}

\section{\faCogs\ Skills \& Interests}
\begin{itemize}[parsep=0.5ex]
  \item \textbf{Bioinformatics Toolkit:} GATK, SAMtools, BCFtools, VCFtools, BEDtools, PLINK, mapDamage, ANNOVAR, VEP, NCBI Entrez Direct (EDirect), SRA Toolkit
  \item \textbf{Algorithms \& Analytics:} HMM (Hidden Markov Model), Maximum Likelihood Estimation, Bayesian Inference, PCA, t-SNE, Clustering (K-Means/Hierarchical)
  \item \textbf{Deep Learning \& Data Science:} PyTorch, PyTorch Geometric (PyG), TensorFlow, Scikit-learn, Pandas, NumPy, SciPy, NetworkX, Matplotlib, Seaborn
  \item \textbf{Programming Languages:} Python, R, Shell/Bash, PHP, SQL, C/C++ (Basic), \LaTeX
  \item \textbf{Development \& Visualization:} Docker, Git/GitHub, Linux (CentOS/Ubuntu), Adobe Illustrator (Scientific Figures), PyMOL
  \item \textbf{Languages:} Mandarin (Native), English (Fluent; IELTS Speaking 7.0)
\end{itemize}

\section{\faFlask\ Professional Interests}
Multi-Omics Integration \textperiodcentered\ AI-Driven Computational Biology \textperiodcentered\ Graph Neural Networks \textperiodcentered\ Biological LLM-Based Agents

\end{document}
